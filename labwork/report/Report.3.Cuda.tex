\documentclass{article}
\usepackage[utf8]{inputenc}
\usepackage{listings}
\usepackage{graphicx}
\usepackage{epstopdf}
\usepackage{verbatim}
\usepackage[english]{babel}

\title{Report.3.Cuda}
\author{Linh Duong}
\date{October 2018}

\lstset{breaklines=true} 
\begin{document}
\maketitle

\section{Implementation}
In this labwork, the code for labwork 1 is re-implemented using CUDA. The flow of the program contains 4 main parts:

\begin{enumerate}
    \item Host feeds device with data
    \begin{verbatim}
cudaMalloc(&devInput, pixelCount * 3);
	cudaMalloc(&devOutput, pixelCount * 3);
	cudaMemcpy(devInput, inputImage->buffer, pixelCount*3, cudaMemcpyHostToDevice);
    \end{verbatim}
    \item Host asks device to process data
    \begin{verbatim}
grayscale<<<numBlock, blockSize>>>(devInput, devOutput);
    \end{verbatim}
    \item Device processes data in parallel
    \begin{verbatim}
__global__ void grayscale(uchar3 *input, uchar3 *output) {
    	int tid = threadIdx.x + blockIdx.x * blockDim.x;
    	unsigned char g = (input[tid].x + input[tid].y + input[tid].z) / 3;
    	output[tid].z = output[tid].y = output[tid].x = g;
}
    \end{verbatim}
    \item Device returns result
    \begin{verbatim}
cudaMemcpy(outputImage, devOutput, pixelCount*3, cudaMemcpyDeviceToHost);
	cudaFree(devInput);
	cudaFree(devOutput);
    \end{verbatim}
\end{enumerate}

\section{Result}
\begin{verbatim}
    USTH ICT Master 2018, Advanced Programming for HPC.
    Warming up...
    Starting labwork 3
    labwork 3 ellapsed 100.8ms
\end{verbatim}
The program's duration is about 30 times faster than sequential programming.
\end{document}
