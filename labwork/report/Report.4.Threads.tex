\documentclass{article}
\usepackage[utf8]{inputenc}
\usepackage{listings}
\usepackage{graphicx}
\usepackage{epstopdf}
\usepackage{verbatim}
\usepackage[english]{babel}

\title{Report.4.Threads}
\author{Linh Duong}
\date{October 2018}

\lstset{breaklines=true} 
\begin{document}
\maketitle

\section{Implementation}
In this labwork, the code for labwork 3 is re-implemented using 2D blocks. The modified parts are:

\begin{itemize}
    \item The kernel
    \begin{verbatim}
__global__ void grayscale2d(uchar3* input, uchar3* output, int width) {
    int tidx = threadIdx.x + blockIdx.x * blockDim.x;
    int tidy = threadIdx.y + blockIdx.y * blockDim.y;
    int tid = tidx + tidy * width; //gridDim.x*blockDim.x != width
    output[tid].x = (input[tid].x + input[tid].y + input[tid].z) / 3;
    output[tid].z = output[tid].y = output[tid].x;
}
    \end{verbatim}
    \item BlockSize and GridSize
    \begin{verbatim}
dim3 blockSize = dim3(32, 32);
dim3 gridSize = dim3(inputImage->width / blockSize.x,
                inputImage->height / blockSize.y);
    \end{verbatim}
\end{itemize}

\section{Result}
The results with different 2D block size values:
\begin{table}[h]
\centering
\begin{tabular}{|c|c|c|c|c|c|c|c|c|c|c|}
\hline
Block size &32x32 &64x16 &16x64 & 16x16  & 32x16  & 16x32 & 64x4 & 4x64 & 1x1024 & 1024x1 \\ \hline
Time (ms)  & 96.3 & 103.9 & 102.2 & 104.1 & 101.8   & 103.9   & 103.7  & 103.2  & 107.5 & 103.7  \\
\hline
\end{tabular}
\end{table}
\end{document}
